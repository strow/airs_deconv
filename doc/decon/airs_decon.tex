\documentclass[11pt]{article}
\title{AIRS Deconvolution \\ 
  {***} DRAFT {***}}
\author{H.~Motteler, L.~Strow, S.~Hannon}
\date{\today}

% acronyms for text or math mode
\newcommand {\ccast} {\mbox{\small CCAST}}
\newcommand {\cris} {\mbox{\small CrIS}}

\newcommand {\airs} {\mbox{\small AIRS}}
\newcommand {\iasi} {\mbox{\small IASI}}
\newcommand {\idps} {\mbox{\small IDPS}}
\newcommand {\nasa} {\mbox{\small NASA}}
\newcommand {\noaa} {\mbox{\small NOAA}}
\newcommand {\nstar} {\mbox{\small STAR}}
\newcommand {\umbc} {\mbox{\small UMBC}}
\newcommand {\uw}   {\mbox{\small UW}}

\newcommand {\fft}  {\mbox{\small FFT}}
\newcommand {\ifft} {\mbox{\small IFFT}}
\newcommand {\fir}  {\mbox{\small FIR}}
\newcommand {\fov}  {\mbox{\small FOV}}
\newcommand {\for}  {\mbox{\small FOR}}
\newcommand {\ict}  {\mbox{\small ICT}}
\newcommand {\ils}  {\mbox{\small ILS}}
\newcommand {\igm}  {\mbox{\small IGM}}
\newcommand {\opd}  {\mbox{\small OPD}}
\newcommand {\rms}  {\mbox{\small RMS}}
\newcommand {\zpd}  {\mbox{\small ZPD}}
\newcommand {\ppm}  {\mbox{\small PPM}}
\newcommand {\srf}  {\mbox{\small SRF}}
\newcommand {\sdr}  {\mbox{\small SDR}}

\newcommand {\ES} {\mbox{\small ES}}
\newcommand {\SP} {\mbox{\small SP}}
\newcommand {\IT} {\mbox{\small IT}}
\newcommand {\SA} {\mbox{\small SA}}

\newcommand {\ET} {\mbox{\small ET}}
\newcommand {\FT} {\mbox{\small FT}}

% abbreviations, mainly for math mode
\newcommand {\real} {\mbox{real}}
\newcommand {\imag} {\mbox{imag}}
\newcommand {\atan} {\mbox{atan}}
\newcommand {\obs}  {\mbox{obs}}
\newcommand {\calc} {\mbox{calc}}
\newcommand {\sinc} {\mbox{sinc}}
\newcommand {\psinc} {\mbox{psinc}}
\newcommand {\std} {\mbox{std}}

% symbols, for math mode only
\newcommand {\wn} {\mbox{cm$^{-1}$}}
\newcommand {\lmax} {L_{\mbox{\tiny max}}}
\newcommand {\vmax} {V_{\mbox{\tiny max}}}

\newcommand {\tauobs} {\tau_{\mbox{\tiny obs}}}
\newcommand {\taucal} {\tau_{\mbox{\tiny calc}}}
\newcommand {\Vdc}  {V_{\mbox{\tiny DC}}}

\newcommand {\rIT} {r_{\mbox{\tiny\textsc{ict}}}}
\newcommand {\rES} {r_{\mbox{\tiny\textsc{es}}}}
\newcommand {\robs} {r_{\mbox{\tiny obs}}}

\newcommand {\rITobs} {r_{\mbox{\tiny\textsc{ict}}}^{\mbox{\tiny obs}}}
\newcommand {\rITcal} {r_{\mbox{\tiny\textsc{ict}}}^{\mbox{\tiny cal}}}

\newcommand {\ITmean} {\langle\mbox{\small IT}\rangle}
\newcommand {\SPmean} {\langle\mbox{\small SP}\rangle}

\begin{document}
\maketitle

\section{AIRS spectral response functions}

The \AIRS\ spectral response functions model instrument channel
response as a function of frequency and associate channels with
nominal center frequencies[1].  Each \AIRS\ channel $i$ has an
associated spectral response function or {\SRF} $\srf_i(v)$ such
that the channel radiance $c_i = \int \srf_i(v)r(v)\,dv$.  The
center or peak of this function is the nominal channel frequency.

Suppose we have $n$ channels and a frequency grid $v$ of $k$ points
spanning the domains of the functions $\srf_i$.  The grid step size
for our applications is often $0.0025\ \wnum$, the kcarta[2]
resolution.  Let $S_k$ be an $n\times k$ array such that $s_{i,j} =
\srf_i(v_j)/w_i$, where $w_i = \sum_j \srf_i(v_j)$, that is where
row $i$ is $\srf_i(v)$ tabulated at the grid $v$ and normalized so
the row sum is 1.  If the channel centers are in increasing order
$S_k$ is banded, and if they are not too close the rows are linearly
independent.  $S_k$ is a linear transform whose domain is radiance
at the grid $v$, and whose range is channel radiances.  If $r$ is
radiance at the grid $v$, then $c = S_k r$ gives a good
approximation of the channel radiances $c_i = \int\srf_i(v)r(v)\,dv$.

In practice this is how we convolve kcarta or other simulated
radiances to get {\AIRS} channel radiances, and transmittances 
for our fast transmittance models[3].  We construct $S_k$ either
explicitly or implicitly from {\AIRS} {\SRF} tabulations.  The
matrix $S_k$ in the former case is large but managable with a banded 
representation or as a Matlab sparse array.

Suppose we have $S_k$ and channel radiances $c$ and want to find
$r$, that is, to deconvolve $c$.  Consider the linear system $S_k x
= c$.  Since $n < k$ for the kcarta grid mentioned above, it is
underdetermined, with infinitely many solutions.  We could add
constraints, take a pseudo-inverse, consider a new matrix $S_b$ with
columns tabulated at some coarser grid, or some combination of the
above.  We begin with two cases, $S_a$ with SRFs tabulated at the
{\AIRS} channel grid, and $S_b$ with SRFs at an intermediate grid,
typically $0.1\ \wnum$, the approximate resolution of the original
measurements.  We then consider the effect of choice of basis set
for the intermdiate representation.

\section{Deconvolution to the AIRS channel grid}

Let $v_a = v_1,v_2,\ldots,v_n$ be channel center frequencies
associated with a set of SRFs.  Similar to $S_k$, let $S_a$ be an
$n\times n$ array where row $i$ is $\srf_i(v)$ tabulated at the $v_a$
grid, with rows normalized to 1.  If the channel centers are in
increasing order $S_a$ is banded, and if they are not too close the
rows are linearly independent.  $S_a$ is a linear transform whose
domain is radiance at the grid $v_a$ and whose range is channel
radiances.  If $r$ is radiance at the grid $v_a$, then $c = S_a r$ is
still an approximation of $\int\srf_i(v)r(v)\,dv$, though not as good
an approximation as for the high-resolution case.

Consider the linear system $S_a x = c$, similar to the case $S_k x =
c$ above, where we are given $S_a$ and channel signals $c$ and want
to find radiances $x$.  Since $S_a$ is an $n \times n$ matrix and
may have linearly independent rows, we might hope to solve for $r$.
However in practice there are problems.  If we take $v_a$ as the
standard AIRS L1b channel set, we find $S_a$ is poorly conditioned,
without a usable inverse.  This is due to the L1b channel spacing,
which because of module overlaps is quite variable, with the closest
channels only $0.0036\ \wnum$ apart.

If we use the the linear function $g(v) = 4\cdot 10^{-4} \cdot v -
0.04$, where $v$ is frequency, as a lower bound on the acceptable
channel spacing, we drop about 64 out of the 2378 L1b channels and
the condition number of $S_a$ is much improved, to around 30.  With
the partly synthetic L1c channel set, $g(v)$ drops only 4 channels
and $\hbox{cond}(S_a)$ is about 250, still low enough for a useful
inverse.

\begin{itemize}
  \item show figures for channel trimming
  \item compare this approach with more recent results
  \item the main point of this section is probably the channel
    trimming and discussion of condition number
\end{itemize}


\section{Deconvolution to the SRF tabulation grid}

We now consider deconvolution to an intermediate grid.  Typically
this will be at $0.1\ \wnum$, the resolution of the tabulated SRFs.
Let $v_b = v_1,v_2,\ldots,v_m$ be channel center frequencies
associated with a set of SRFs.  Similar to $S_k$, let $S_b$ be an
$n\times m$ array where row $i$ is $\srf_i(v)$ tabulated at the
$v_m$ grid, with rows normalized to 1.  If the channel centers are
in increasing order $S_m$ is banded, and if they are not too close
the rows are linearly independent.  $S_m$ is a linear transform
whose domain is radiance at the grid $v_m$ and whose range is
channel radiances.  If $r$ is radiance at the grid $v_b$, then $c =
S_b r$ is a good approximation of $\int\srf_i(v)r(v)\,dv$.

Consider the linear system $S_b x = c$, similar to the case $S_k x =
c$ above, where we are given $S_b$ and channel signals $c$ and want
to find radiances $x$.  Since $n < m < k$, as with $S_k$, the system
will be underdetermined, but more manageable because for the default
resolutions $m$ is approximately 40 times less than $k$, so finding
the pseudo-inverse of $S_b$ becomes tractible.

\begin{itemize}
  \item needs figures from 
  \item add recent blog reports and results from cris\_test4
  \item there is some redundant detail in the descriptions of
    $S_{a,b,k}$ but that can stay until the final form is clear.
\end{itemize}

\section{A sinc basis for the intermediate representation}

\section{Applications}

\end{document}

